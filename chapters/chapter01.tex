\chapter{萌新教程}
\section{这是中标题}
emmmm
\subsection{这是小标题}
emmmmm
\subsubsection{这是小小标题}
搞这么多层大丈夫?

\section{表格}

使用 \href{http://www.tablesgenerator.com/}{http://www.tablesgenerator.com/} 生成, 可粘贴Excel.

\begin{table}[!h]
    \centering
    \caption{My caption}
    \label{my-label}
    \begin{tabular}{@{}llll@{}}
        \toprule
        $A$ & $B$ & $A+B$ & $A\times B$ \\ \midrule
        1   & 6   & 7     & 6           \\
        2   & 7   & 9     & 14          \\
        3   & 8   & 11    & 24          \\
        4   & 9   & 13    & 36          \\
        5   & 10  & 15    & 50          \\ \bottomrule
    \end{tabular}
\end{table}

\section{特殊符号}

用 \href{http://detexify.kirelabs.org/classify.html}{http://detexify.kirelabs.org/classify.html}
画出来.

\section{参考文献的引用}

\LaTeX{} 中要求参考文献使用 \lstinline`\cite` 进行参考引用, 但是由于论文要求中说明需在
文字的右上角注明引用, 所以请使用预定义好的命令 \lstinline`\ucite` 进行参考引用. 比如本论
文模板 `LaTeX-Template-For-NPU-Thesis' \ucite{NWPUThesisLaTeXTemplate} 要求务必
声明引用, 同时预配置了插件 `math-symbols' \ucite{MathSymbolsinLaTeXbypolossk}. 对
组件的引用是每一名科学工作者的基本素养(一本正经). 对于需要引用但是并不需要明确指明引用位置
的文献, 请使用 \lstinline`\nocite` 命令.

在此同时感谢真正的 dalao 高德纳开发了全世界版本号最接近 $\pi$ 的软件 \LaTeX{}
\ucite{knuth1986the}\nocite{lamport1989latex:}.


\endinput
