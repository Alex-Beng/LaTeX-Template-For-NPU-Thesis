%------------------------------------------------------------------------------%
%                                                                              %
%   LaTeX Template for Bachlor Thesis of Northwestern Polytechnical University %
%   Using XeLeTeX + MakeIndex + BibTeX, or Using CTeX v2.9.2.164               %
%   Version: 1.3.0                                                             %
%                                                                              %
%------------------------------------------------------------------------------%
%   Copyright 2016 by Shangkun Shen, MIT-LICENSE(see mit-license.polossk.com)  %
%------------------------------------------------------------------------------%


%---------------------------------纸张大小设置---------------------------------%
\usepackage{geometry}
% 普通A4格式缩进
% \geometry{left=2.5cm,right=2.5cm,top=2.5cm,bottom=2.5cm}
% 论文标准缩进
\geometry{left=1.25in,right=1.25in,top=1in,bottom=1.5in}
%------------------------------------------------------------------------------%


%----------------------------------必要库支持----------------------------------%

\usepackage{xcolor}
\usepackage{tikz}
\usepackage{layouts}
\usepackage[numbers,sort&compress]{natbib}
\usepackage{clrscode}
\usepackage{gensymb}
\usepackage[final]{pdfpages}
%------------------------------------------------------------------------------%


%--------------------------------设置标题与目录--------------------------------%
\usepackage[sf]{titlesec}
\usepackage{titletoc}
%------------------------------------------------------------------------------%


%--------------------------------添加书签超链接--------------------------------%
\usepackage[unicode=true,colorlinks=false,pdfborder={0 0 0}]{hyperref}
% 在此处修改打开文件操作
\hypersetup{
    bookmarks=true,         % show bookmarks bar?
    pdftoolbar=true,        % show Acrobat’s toolbar?
    pdfmenubar=true,        % show Acrobat’s menu?
    pdffitwindow=true,      % window fit to page when opened
    pdfstartview={FitH},    % fits the width of the page to the window
    pdfnewwindow=true,      % links in new PDF window
}
% 在此处添加文章基础信息
\hypersetup{
    pdftitle={title},
    pdfauthor={author},
    pdfsubject={subject},
    pdfcreator={creator},
    pdfproducer={producer},
    pdfkeywords={key1  key2  key3}
}
%------------------------------------------------------------------------------%


%---------------------------------设置字体大小---------------------------------%
\usepackage{type1cm}
% 字号与行距,统一前缀s(a.k.a size)
\newcommand{\sChuhao}{\fontsize{42pt}{63pt}\selectfont}         		% 初号, 1.5倍
\newcommand{\sYihao}{\fontsize{26pt}{36pt}\selectfont}          		% 一号, 1.4倍
\newcommand{\sErhao}{\fontsize{22pt}{28pt}\selectfont}          		% 二号, 1.25倍
\newcommand{\sXiaoer}{\fontsize{18pt}{18pt}\selectfont}         		% 小二, 单倍
\newcommand{\sSanhao}{\fontsize{16pt}{24pt}\selectfont}         		% 三号, 1.5倍
\newcommand{\sXiaosan}{\fontsize{15pt}{22pt}\selectfont}        		% 小三, 1.5倍
\newcommand{\sSihao}{\fontsize{14pt}{21pt}\selectfont}          		% 四号, 1.5倍
\newcommand{\sSHalfXiaosi}{\fontsize{13pt}{19pt}\selectfont}  			% 半小四, 1.5倍
\newcommand{\sHalfXiaosi}{\fontsize{13pt}{16.25pt}\selectfont}  		% 半小四, 1.25倍
\newcommand{\sRealHalfXiaosi}{\fontsize{12.5pt}{16.25pt}\selectfont}  	% 模板中的半小四, 1.25倍
\newcommand{\sXiaosi}{\fontsize{12pt}{14.4pt}\selectfont}       		% 小四, 1.25倍
\newcommand{\sLargeWuhao}{\fontsize{11pt}{11pt}\selectfont}     		% 大五, 单倍
\newcommand{\sWuhao}{\fontsize{10.5pt}{10.5pt}\selectfont}      		% 五号, 单倍
\newcommand{\sXiaowu}{\fontsize{9pt}{9pt}\selectfont}           		% 小五, 单倍
%------------------------------------------------------------------------------%


%---------------------------------设置中文字体---------------------------------%
\usepackage{fontspec}
%\usepackage[SlantFont,BoldFont,CJKchecksingle,CJKnumber]{xeCJK}
\usepackage[SlantFont,BoldFont,CJKchecksingle]{xeCJK}
% 使用 Adobe 字体
\newcommand\adobeSog{Adobe Song Std}
\newcommand\adobeHei{Adobe Heiti Std}
\newcommand\adobeKai{Adobe Kaiti Std}
\newcommand\adobeFag{Adobe Fangsong Std}
\newcommand\codeFont{Consolas}
% 设置字体
\defaultfontfeatures{Mapping=tex-text}
\setCJKmainfont[ItalicFont=\adobeKai, BoldFont=\adobeHei]{\adobeSog}
\setCJKsansfont[ItalicFont=\adobeKai, BoldFont=\adobeHei]{\adobeSog}
\setCJKmonofont{\codeFont}
\setmonofont{\codeFont}
% 设置字体族
\setCJKfamilyfont{song}{\adobeSog}      % 宋体  
\setCJKfamilyfont{hei}{\adobeHei}       % 黑体  
\setCJKfamilyfont{kai}{\adobeKai}       % 楷体  
\setCJKfamilyfont{fang}{\adobeFag}      % 仿宋体
% 用于页眉学校名,特殊字体,powerby https://github.com/ecomfe/fonteditor
\setCJKfamilyfont{nwpu}{nwpuname}
% 新建字体命令,统一前缀f(a.k.a font)
\newcommand{\fSong}{\CJKfamily{song}}
\newcommand{\fHei}{\CJKfamily{hei}}
\newcommand{\fFang}{\CJKfamily{fang}}
\newcommand{\fKai}{\CJKfamily{kai}}
\newcommand{\fNWPU}{\CJKfamily{nwpu}}
%------------------------------------------------------------------------------%


%------------------------------添加插图与表格控制------------------------------%
\usepackage{graphicx}
\usepackage[font=small,labelsep=quad]{caption}
\usepackage{wrapfig}
\usepackage{multirow,makecell}
\usepackage{longtable}
\usepackage{booktabs}
\usepackage{tabularx}
\usepackage{setspace}
%------------------------------------------------------------------------------%


%---------------------------------添加列表控制---------------------------------%
\usepackage{enumerate}
\usepackage{enumitem}
%------------------------------------------------------------------------------%


%---------------------------------设置引用格式---------------------------------%
\renewcommand\figureautorefname{图}
\renewcommand\tableautorefname{表}
\renewcommand\equationautorefname{式}
\newcommand\myreference[1]{[\ref{#1}]}
\newcommand\eqrefe[1]{式(\ref{#1})}
\renewcommand\theequation{\thechapter-\arabic{equation}}
% 增加 \ucite 命令使显示的引用为上标形式
\newcommand{\ucite}[1]{$^{\mbox{\scriptsize \cite{#1}}}$}
%------------------------------------------------------------------------------%


%--------------------------------设置定理类环境--------------------------------%
\usepackage[amsthm,thmmarks]{ntheorem}
\newtheorem{myexample}{例}
\newtheorem{thm}{定理}
%------------------------------------------------------------------------------%


%--------------------------设置中文段落缩进与正文版式--------------------------%
\XeTeXlinebreaklocale "zh"       %使用中文的换行风格
\XeTeXlinebreakskip = 0pt plus 1pt    %调整换行逻辑的弹性大小
% \xeCJKcaption{gb_452}
\usepackage{indentfirst}
\setlength{\parindent}{26pt}
\setlength{\parskip}{3pt plus 1pt minus 1pt} % 段落间距
\renewcommand{\baselinestretch}{1.25} % 行距
%------------------------------------------------------------------------------%


%----------------------------设置段落标题与目录格式----------------------------%
\setcounter{secnumdepth}{3}
\setcounter{tocdepth}{2}
\usepackage{CJKnumb}

% 正文中标题格式,毋需标号
% \titleformat{\section}[hang]{\fHei \sf \sSihao}
%     {\sSihao }{0.5em}{}{}
% \titleformat{\subsection}[hang]{\fHei \sf \sHalfXiaosi}
%     {\sHalfXiaosi }{0.5em}{}{}
% \titleformat{\subsubsection}[hang]{\fHei \sf}
%     {\thesubsubsection }{0.5em}{}{}
% 正文中标题格式,需要标号

\newcommand\chapterID[1]{第\CJKnumber{#1}章}
\renewcommand{\chaptername}{第~\CJKnumber{\thechapter}~章}
%\newcommand\chapterID[1]{第\zhnumber{#1}章}
%\renewcommand{\chaptername}{第~\zhnumber{\thechapter}~章}
%\renewcommand{\chaptername}{第 \thechapter 章}
\renewcommand{\figurename}{图}
\renewcommand{\tablename}{表}
\renewcommand{\bibname}{参考文献}
\renewcommand{\contentsname}{目~录}
\newcommand{\keywords}[1]{\\ \\ \textbf{关~键~词}:#1}


\titleformat{\chapter}[hang]{\normalfont\sSanhao\filcenter\fHei\bf}%
{\sSanhao{\chaptertitlename}}{20pt}{\sSanhao}
\titleformat{\section}[hang]{\fHei \bf \sXiaosan}%
{\sXiaosan \thesection}{0.5em}{}{}
\titleformat{\subsection}[hang]{\fHei \bf \sSHalfXiaosi}%
{\sSHalfXiaosi \thesubsection}{0.5em}{}{}
%\titleformat{\subsubsection}[hang]{\fHei \bf}%		%普通的subsubsection:1.2.3.4 标题
%    {\thesubsubsection }{0.5em}{}{}
\titleformat{\subsubsection}[hang]{\fHei \bf}		%小标题式的subsubsection:(4) 标题
{(\arabic{subsubsection})}{0.5em}{}{}


% 缩小正文中各级标题之间的缩进
\titlespacing{\chapter}{0pt}{-3ex plus .1ex minus .2ex}{0.25em}
\titlespacing{\section}{0pt}{-0.2em}{0em}
\titlespacing{\subsection}{0pt}{0.5em}{0em}
\titlespacing{\subsubsection}{0pt}{0.25em}{0pt}

% 定义目录中各级标题之间的格式以及缩进
% \dottedcontents{chapter}[0.0em]{\fHei\vspace{0.5em}}{0.0em}{5pt}
\dottedcontents{section}[1.16cm]{}{1.8em}{5pt}
\dottedcontents{subsection}[2.00cm]{}{2.7em}{5pt}
\dottedcontents{subsubsection}[2.86cm]{}{3.4em}{5pt}
\titlecontents{chapter}[0pt]{\fHei\vspace{0.5em}}%
{\contentsmargin{0pt}\fHei\makebox[0pt][l]{\chapterID{\thecontentslabel}}\hspace{3.8em}}%
{\contentsmargin{0pt}\fHei}%
{\titlerule*[.5pc]{.}\contentspage}[\vspace{0em}]
%------------------------------------------------------------------------------%


%---------------------------------设置页眉页脚---------------------------------%
\usepackage{fancyhdr}
\usepackage{fancyref}
%\addtolength{\headsep}{-0.1cm}          %页眉位置
%\addtolength{\footskip}{-0.1cm}         %页脚位置
\addtolength{\topmargin}{0.5cm}
\newcommand{\makeheadrule}{
    \makebox[0pt][l]{\rule[.7\baselineskip]{\headwidth}{0.8pt}}
    \vskip-.8\baselineskip
}
\makeatletter
\renewcommand{\headrule}{%
    {
            \if@fancyplain\let\headrulewidth\plainheadrulewidth\fi
            \makeheadrule
        }
}
\pagestyle{fancyplain}
\fancyhf{}
\fancyfoot[C,C]{\sWuhao-~\thepage~-}
% 后续文字可以自行修改
\chead{\sSanhao\raisebox{0.04cm}{ \fNWPU 西北工业大学} \fSong{{\textbf{本科毕业设计论文} }}}
%------------------------------------------------------------------------------%


%----------------------------------其他补充设置--------------------------------%
% 重置列表环境的间隔
% \let\orig@Itemize =\itemize
% \let\orig@Enumerate =\enumerate
% \let\orig@Description =\description

% \def\Myspacing{
%     \itemsep=1.5ex \topsep=-0.5ex \partopsep=0pt \parskip=0pt \parsep=0.5ex
% }

% \def\newitemsep{
%     \renewenvironment{itemize}{\orig@Itemize\Myspacing}{\endlist}
%     \renewenvironment{enumerate}{\orig@Enumerate\Myspacing}{\endlist}
%     \renewenvironment{description}{\orig@Description\Myspacing}{\endlist}
% }

% \def\olditemsep{
%     \renewenvironment{itemize}{\orig@Itemize}{\endlist}
%     \renewenvironment{enumerate}{\orig@Enumerate}{\endlist}
%     \renewenvironment{description}{\orig@Description}{\endlist}
% }

% \newitemsep
% 下划线
\newcommand\dlmu@underline[2][5cm]{\hskip1pt\underline{\hb@xt@ #1{\hss#2\hss}}\hskip3pt}
\let\coverunderline\dlmu@underline
%------------------------------------------------------------------------------%



%----------------------------------添加代码控制--------------------------------%
\usepackage{listings}
\lstset{
    basicstyle=\footnotesize\ttfamily,
    %numbers=left,
    %numberstyle=\tiny,
    %numbersep=5pt,
    tabsize=4,
    extendedchars=true,
    breaklines=true,
    keywordstyle=\color{blue},
    numberstyle=\color{purple},
    commentstyle=\color{olive},
    stringstyle=\color{orange}\ttfamily,
    showspaces=false,
    showtabs=false,
    framexrightmargin=5pt,
    framexbottommargin=4pt,
    showstringspaces=false
    escapeinside=`', %逃逸字符(1左面的键),用于显示中文
}
\renewcommand{\lstlistingname}{CODE}
\lstloadlanguages{% Check Dokumentation for further languages, page 12
    Pascal, C++, Java, Ruby, Python, Matlab, R
}
%------------------------------------------------------------------------------%

\renewcommand\arraystretch{1.4}
\renewcommand{\thefigure}{\thechapter-\arabic{figure}}
\renewcommand{\thetable}{\thechapter-\arabic{table}}
\captionsetup[table]{labelfont=bf,textfont=bf}
\graphicspath{{figures/}}

\endinput
% 这是简单的 thesis(article) 的导言区设置,不能单独编译。
